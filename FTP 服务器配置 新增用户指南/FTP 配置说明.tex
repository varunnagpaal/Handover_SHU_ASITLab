\documentclass{article}
\usepackage[hidelinks]{hyperref}
\usepackage{xeCJK}
\usepackage{indentfirst} 
\usepackage{geometry}
\geometry{a4paper,top=2.54cm,bottom=2.54cm,left=3.18cm,right=3.18cm}
\parindent 2em % 段首缩进
\pagenumbering{roman}

\begin{document}

\author{林庆毫}
\title{FTP 配置说明}
\maketitle
\tableofcontents
\newpage
\pagenumbering{arabic}

\section{FTP 连接方法}
\begin{enumerate}
	\item 连接键盘、鼠标、屏幕使用。

	\item Windows 远程桌面连接

	      连接实验室网络 ASITLAB\_1,打开 Windows 远程桌面连接,按下图填写计算机 192.168.2.254,用户名 Administrator,连接密码为 asitlab。
	      \begin{figure}[!h]
		      \centering
		      \includegraphics[scale=0.3]{../assets/images/2021-06-23_18-10-47.png}
	      \end{figure}
\end{enumerate}

\section{FTP 服务端方案}
使用 FileZilla Server 软件搭建。

\section{FTP 目录说明}
FTP 根目录位于服务器 C 盘的 FTP\_storage。

temp 目录:临时交互用。

pub 目录:主要目录,其中 asitlab 子文件内含各用户自己管理的文档。

incoming 目录:结构与 pub 类似。

\section{FTP 用户添加}
用户统一以“名的首字母+姓的全拼”组成。

打开 FTP 桌面上的 FileZilla Server Interface,按下图操作。
\begin{figure}[!h]
	\centering
	\includegraphics[scale=0.6]{../assets/images/2021-06-23_18-22-27.png}
\end{figure}


\end{document}