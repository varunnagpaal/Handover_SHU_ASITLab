\documentclass{article}
\usepackage[hidelinks]{hyperref}
\usepackage{xeCJK}
\usepackage{indentfirst} 
\usepackage{geometry}
\geometry{a4paper,top=2.54cm,bottom=2.54cm,left=3.18cm,right=3.18cm}
\parindent 2em % 段首缩进
\pagenumbering{roman}

\begin{document}

\author{林庆毫}
\title{开发服务器的相关说明}
\maketitle
\tableofcontents
\newpage
\pagenumbering{arabic}

\section{开发软件的安装路径}
在开发服务器的 ServerB 上,有一通过 NFS 共享的 /export 目录,四台服务器都可访问。该目录下安装开发所需的各类软件,如 Matlab、Xilinx 系列软件、Cadence 系列软件、Synopsys 系列软件。各类软件都安装在 ServerB 上导致其存储空间紧张,因此勿在 ServerB 上进行开发工作。

\section{软件的配置脚本}
软件的配置通过用户各自的\ .bashrc 脚本完成。/export/Wares/README/ 目录下有说明文档和一份\ .bashrc 样例供参考。

几乎所有的软件配置脚本都存放在 /export/Wares/sh/ 目录下。

\section{软件的证书}
所有的软件证书都存放在 /export/Wares/sh/lic/ 目录下。

\section{用户组与用户的管理}
服务器的用户组有 asitlab/asitlab\_a/asitlab\_b 等,但目前其实并无严格的分组规划,一般来说将用户加入 asitlab 组即可。

用户统一以“名的首字母+姓的全拼”组成,添加新用户的方法可参考 /export/project/unify/qhlin/miscellaneous/useradd\_script.sh 脚本。

\section{停电时需要进行的操作}
建议设置定时提醒,每周查看一次学校官网通告,确认是否有停电计划。

停电前在群内通知老师、同学,提醒提前保存工作结果。停电前 30 分钟关闭所有 4 台开发服务器,由于服务器存在一些不明问题,输入关机命令后可能进入“ssh 无法再连接上,但服务器仍未关机且一直能 ping 通”的状态。因此建议在输入关机命令前强制 kill 绝大部分正在运行的程序,可以降低发生问题的概率。

另外建议最后关闭 ServerB,因其存在可能影响其他服务器的 NFS 进程。

\section{开发服务器存在的问题}
ServerB 与 D (A 与 C 未验证过),在连接屏幕用键鼠直接操作时,从屏幕上看会卡在开机进度条界面,但此时可通过 ssh 和 X11 等服务正常连接,且能看到图形界面。问题产生原因不明。

\end{document}