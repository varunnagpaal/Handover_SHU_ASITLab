\documentclass{article}
\usepackage[hidelinks]{hyperref}
\usepackage{xeCJK}
\usepackage{indentfirst} 
\usepackage{geometry}
\usepackage{framed}
\geometry{a4paper,top=2.54cm,bottom=2.54cm,left=3.18cm,right=3.18cm}
\parindent 2em % 段首缩进
\pagenumbering{roman}

\begin{document}

\author{林庆毫}
\title{打印机的相关说明}
\maketitle
\tableofcontents
\newpage
\pagenumbering{arabic}

\section{打印机所处网络环境}
无线连接 SHU\_ASITLAB1。

IP 固定为 192.168.2.253。

192.168.2.253 的 9100 端口映射到路由器 WAN 口的 9100 端口。

\section{驱动}
通常情况下,在 SHU\_ASITLAB1 网络内使用打印机无须手动安装驱动。在系统设置中完成添加打印机时将自动完成驱动的安装。

部分旧电脑或有在校园网范围内访问打印机的需求,需要手动安装一次驱动。

\url{https://support.hp.com/cn-zh/drivers/selfservice/hp-laserjet-pro-mfp-m128-series/5396667/model/5303429}

\section{校园网内使用打印机的方式}
\url{https://www.huaweicloud.com/articles/b5ba8ddefbfba625604748110a695483.html}

将具体步骤中的“打印机名或 IP 地址”更换为路由器 WAN 口 IP。

\section{扫描}
打印机因不明原因无法正常通过无线网络扫描,需要使用 USB 与电脑进行有线连接再扫描。

\end{document}