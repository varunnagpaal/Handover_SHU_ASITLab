\documentclass{article}
\usepackage[hidelinks]{hyperref}
\usepackage{xeCJK}
\usepackage{indentfirst} 
\usepackage{geometry}
\geometry{a4paper,top=2.54cm,bottom=2.54cm,left=3.18cm,right=3.18cm}
\parindent 2em % 段首缩进
\pagenumbering{roman}

\begin{document}

\author{林庆毫}
\title{实验室路由器、网络}
\maketitle
\tableofcontents
\newpage
\pagenumbering{arabic}

\section{路由器管理地址、密码}
\begin{itemize}
    \item 管理地址:192.168.2.1
    \item 管理密码:asitlab
\end{itemize}

\section{WiFi}
\begin{itemize}
    \item 接入点:SHU\_ASITLAB1
    \item WiFi 密码:asitlab2018
    \item 访客网络接入点:SHU\_ASITLAB2
    \item 访客网络 WiFi 密码:2018\_asitlab
          
          访客网络指网络内的设备无法互相访问,例如无法使用打印机、FTP。一般不使用。
\end{itemize}


\section{固定 IP 的设备}
\begin{itemize}
    \item FTP:192.168.2.254
    \item 打印机:192.168.2.253
\end{itemize}

\section{路由器端口映射}
\begin{itemize}
    \item 外部 21 端口映射到 FTP 21 端口
          
          使在校园网内能通过 WAN IP 访问 FTP

    \item 外部 3389 端口映射到 FTP 3389 端口
          
          方便使用 Windows 远程桌面连接 FTP 进行配置、调试
          
    \item 外部 9100 端口映射到打印机 9100 端口
          
          使在校园网内能通过 WAN IP 使用打印机(仅可打印)
\end{itemize}

\section{实验室内网络接口}
B523 靠门东侧墙脚有 4 个接入校园网的以太网口,供路由器和不带无线网卡的主机使用。房间内其余位置的网口,如南侧墙脚,都只是东侧引出的延长线,需要在东侧接入对应编号的网线才能使用。

\end{document}