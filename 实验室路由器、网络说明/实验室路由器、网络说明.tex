\documentclass{article}
\usepackage[hidelinks]{hyperref}
\usepackage{xeCJK}
\usepackage{indentfirst} 
\usepackage{geometry}
\geometry{a4paper,top=2.54cm,bottom=2.54cm,left=3.18cm,right=3.18cm}
\parindent 2em % 段首缩进
\pagenumbering{roman}

\begin{document}

\author{林庆毫}
\title{实验室路由器、网络}
\maketitle
\tableofcontents
\newpage
\pagenumbering{arabic}

\section{路由器管理地址、密码}
\begin{itemize}
    \item 地址:192.168.2.1
    \item 密码:asitlab
\end{itemize}

\section{固定 IP 的设备}
\begin{itemize}
    \item FTP:192.168.2.254
    \item 打印机:192.168.2.253
\end{itemize}

\section{路由器的外部端口映射}
\begin{itemize}
    \item 外部 21 端口映射到 FTP 21 端口
          
          使在校园网内能通过 WAN IP 访问 FTP
          
    \item 外部 9100 端口映射到打印机 9100 端口
          
          使在校园网内能通过 WAN IP 使用打印机(仅可打印)
\end{itemize}

\end{document}